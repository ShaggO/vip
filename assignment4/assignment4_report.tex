\documentclass[11pt,a4paper]{article}

\usepackage[
    top    = 1in,
    right  = 1.5in,
    bottom = 1in,
    left   = 1.5in]{geometry}
\usepackage[utf8]{inputenc}
\usepackage[english]{babel}
\usepackage[T1]{fontenc}
\usepackage{lmodern}

\usepackage{amsmath,amssymb,amsfonts,float,subcaption,mathtools}

\title{Vision and Image Processing\\Assignment 4}
\author{Malte Stær Nissen    \\ \texttt{tgq958} \and
        Benjamin Braithwaite \\ \texttt{cpg608}}

\begin{document}
\maketitle

\section{Initial notes}
%
We have in this assignment implemented our own KLT tracker for tracking patches
throughout video sequences. In this report we first describe how the
tracker is implemented and then test the tracker on a video sequence.
Finally we will comment on the results and shortly discuss ways of improving the
tracker.
%
\section{Kanade-Lucas-Tomasi (KLT) tracker}
%
As explained above we have implemented a KLT tracker. The tracker is built for
tracking fixed sized patches in video sequences. The main idea behind the
tracker algorithm is to estimate a displacement vector for each of the patches we
wish to track. We first assume the image change to be linear making us able to
model the image by a truncated Taylor series using only first order information.

Given a patch we now want to be able to compute the image displacement which
minimizes the squared intensity difference at the patch position in two
subsequent images. This is done by differentiating the residue expression which
leads us to compute a $2\times2$ matrix $G$ and the vector $\boldsymbol{e}$ being the residue
projected onto the gradient for each patch in the image.
\begin{align*}
    G &\approx \sum_{x \in W}
        \begin{bmatrix}
                g_x^2(x)      & g_x(x) g_y(x) \\
                g_x(x) g_y(x) & g_y^2(x)
        \end{bmatrix} w(x) \\
    e &\approx \sum_{x \in W}
    \begin{bmatrix}
        (I(x) - J(x)) g_x(x) w(x) \\
        (I(x) - J(x)) g_y(x) w(x)
    \end{bmatrix}
\end{align*}
Here $I$ is the current image and $J$ is the next image, and $g_x$ and
$g_y$ are the first order derivatives of $I$. We chose $w(x) = 1$ for
simplicity and we utilize the first order of the Gaussian distribution with
standard deviation $\sigma$ to first smooth and then derive our images in one
convolution (one for each direction).

Hereafter we compute the eigenvalues of $G$. Since $G$ is a $2 \times 2$ matrix,
we can compute the eigenvalues by solving the 2nd degree polynomial we
get from using the property of eigenvalues that:
\begin{align*}
    \text{det}\left( \lambda I - G\right) &= 0
\end{align*}
We found this method to be substantially faster than the built-in MATLAB eigenvalue function. Hereafter we filter the patches by only keeping those patches having their
lowest eigenvalue $min(\lambda_1,\lambda_2) > t_{init}$ where $t_{init}$ is the
initial threshold. This ensures that both eigenvalues are
sufficiently large and ensures that the patch contains a corner or texture.
Furthermore we only keep non-overlapping patches. This is done by sorting the patches by their
lowest eigenvalue in descending order and walking through the list from the top
throwing away a patch if it overlaps with one of the patches we already decided
to keep.

This detection of patches (implemented in \texttt{get\_features.m}) could be
performed in several iterations but for simplicity we haven chosen only do it in the start and then to
try to track the initial features for as long as possible.

When we have the filtered features we compute the estimate of the image
displacement of each patch by computing $G^{-1} \boldsymbol{e}$ where $G^{-1}$
is the pseudo-inverse of $G$ in order to avoid computing the inverse of a close
to singular matrix.
Having computed the displacement the first time, we now want to refine the exact
location of the patch in between pixels. We do this by interpolating the image
values using a bilinear interpolation of the coordinates of the new patch. We
do this for $i$ iterations in order to come up with the best estimate of patch
position. In the original KLT algorithm this looping is done until convergence of the residue value, or the patch is thrown away if the patch if doesn't converge. We left out this part due to lack of time.

When performing the computation of the displacement we furthermore throw away
any patches that leaves the image including those who only partically overlap
with the image boundaries.

We now have the new estimates of the positions of our original patches and use
these estimates as the new patch in the next iteration. When running the
algorithm we use the initial threshold $t_{init}$ for thresholding the initial
patch candidates by their lowest eigenvalue. In the following iterations we use
a lower threshold $t_{keep}$ to filter away tracked patches which might have
diverged away from their correct position.
%
\section{Testing}
%
We test our tracker on the provided dudekface image sequence using the best parameters we have found: $\sigma = 2$, 15x15 patch size, initial threshold of $t_{init} = 3 \cdot 10^{-4}$, keep threshold of $t_{keep} = 3 \cdot 10^{-6}$, and 5 convergence iterations. Selected frames with the tracked feature patches and displacement vectors are shown in Figure \ref{fig:tracker_dudekface}.

\begin{figure}
\centering
\begin{subfigure}{0.45\textwidth}
\includegraphics[scale=0.4,trim={120 250 0 250}]{img/tracker_dudekface_1.pdf}
\end{subfigure}
\begin{subfigure}{0.45\textwidth}
\includegraphics[scale=0.4,trim={70 250 0 250}]{img/tracker_dudekface_10.pdf}
\end{subfigure}
\begin{subfigure}{0.45\textwidth}
\includegraphics[scale=0.4,trim={120 250 90 250}]{img/tracker_dudekface_100.pdf}
\end{subfigure}
\begin{subfigure}{0.45\textwidth}
\includegraphics[scale=0.4,trim={70 250 90 250}]{img/tracker_dudekface_213.pdf}
\end{subfigure}
\caption{Frames 1, 10, 100 and 231 from our KLT tracker used on the dudekface image sequence. Feature patches are shown in red and (scaled) displacement vectors for these in blue.}
\label{fig:tracker_dudekface}
\end{figure}

We see that many of the initial features found by the tracker are given a large displacement vector, resulting in them being lost by the thresholding step very shortly after. Apart from this, most of the features stay at the same real-world positions from frame 1 to 10 as desired. A few features such as the one at the speaker to the right move in real-world position, which is a mistake of the tracker.

As time passes more and more of the features are lost, some suddenly for seemlingly no reason, and others through too fast motion or occlusion. For example the hand wave across the face seen in frame 231 causes the remaining face features to be lost. This can be expected as we have not implemented any kind of recovery mechanism, should features not be visible for a short duration.

Overall our KLT tracker performs quite badly regarding tracking accuracy and computation time, taking several seconds to compute some frames. Possible improvements include detection of new patches every few frames and/or some kind of patch matching mechanism between frames to prevent the issue of features changing in real-world position.
%
\end{document}

